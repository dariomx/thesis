\begin{frame}
  \frametitle{The Laplacian opens the door to L.A.}
  \begin{block}{Why the ratio is needed?}
    The problem $\min\limits_{C_1 \ds{\subset} V} \func{cut}(C1,\stcomp{C1})$ can be solved in polynomial time but usually leads to extremely unbalanced partitions (eg \scriptsize
    $\abs{C_1}=1 \land \abs{\stcomp{C1}}=n-1$
    \normalsize).
  \end{block}
  \begin{block}{Balanced but hard}
    The problem $\min\limits_{C_1 \ds{\subset} V} \func{RatioCut}(C1,\stcomp{C1})$ leads to better balanced partitions but is NP-Hard.
  \end{block}
  \begin{block}{Enter the Laplacian (still hard but ...)}
\setlength\abovedisplayskip{0pt}
\[
\min\limits_{C_1 \ds{\subset} V} \func{RatioCut}(C1,\stcomp{C1})
\ds{\equiv} \min\limits_{\vec{f} \in \R{n}}
 \frac{1}{2\abs{V}}\sum\limits_{i=1}^{n} \sum\limits_{j=1}^n w_{ij} (f_i - f_j)^2
\ds{\equiv} \min\limits_{\vec{f} \in \R{n}}
  \underbrace{\trans{\vec{f}} L \ds{\vec{f}}}_{\text{???}}
\]
\[
\setlength\abovedisplayskip{0pt}
\ds{\suchthat}
f_i =
\scriptsize
\begin{cases}
  \sqrt{\abs{\stcomp{C_1}}/\abs{C_1}}, & \text{if } v_i \in C_1 \\
  -\sqrt{\abs{C_1}/\abs{\stcomp{C_1}}}, & \text{if } v_i \in \stcomp{C_1}
\end{cases}
\normalsize
\ds{\ds{\land}}
L = D - W
\ds{\ds{\text{,}}}
\vec{f} \bot \vec{1}
\ds{\land}
\norm{\vec{f}} = \sqrt{n}
\]
  \end{block}    
\end{frame}
