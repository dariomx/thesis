\thispagestyle{plain}
\begin{center}
  %\Large
  %\textbf{Practical Computation of the Fiedler Vector in a Single Processor}

  \vspace{0.4cm}

  %\vspace{0.4cm}
  %\textbf{Darío Bahena Tapia}

  \vspace{0.9cm}
  \textbf{Abstract}
\end{center}

    On the context of a real-life application doing Spectral
    Clustering; the objective of this thesis is to emit an algorithm
    recommendation, for efficient 
    computation of the the Fiedler Vector on a single processor
    (application requirement). The algorithms considered were
    MRRR, Lanczos/IRLM and LOBPCG; the first one is designed for dense
    matrices, hence not really a competitor here; but was 
    included for the sake of comparison (as the application currently
    uses a procedure of such kind). The algorithms were put into
    a competition through a series of tests on a commodity laptop, 
    using matrices generated from the application's domain. Even with
    an optimized Cholesky linear solver, the best known Lanczos/IRLM
    implementation (ARPACK) failed to compete with LOBPCG, which comes
    as the clear winner of the experiment. The results suggest that
    for the biggest matrices considered (size $\approx$ 4500), the
    execution times can 
    move from the minute into the sub-second scale; and to a couple of
    seconds, if we include the time of the pre-processing to eliminate
    clustered eigenvalues. 
