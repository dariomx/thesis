\thispagestyle{plain}
\begin{center}
  %\Large
  %\textbf{Practical Computation of the Fiedler Vector in a Single Processor}

  \vspace{0.4cm}

  %\vspace{0.4cm}
  %\textbf{Darío Bahena Tapia}

  \vspace{0.9cm}
  \textbf{Resumen}
\end{center}

    En el contexto de una aplicación de la vida real que usa Spectral, 
    Clustering; el objetivo de esta tesis es emitir una recomendación
    de algoritmo, para el cálculo eficiente del Fiedler Vector en un
    solo procesador (requerimiento de la aplicación). Los algoritmos
    considerados fueron MRRR, Lanczos/IRLM y LOBPCG; el primero está
    diseñado para matrices densas, así que realmente no es un competidor;
    pero fue incluido para efectos de comparación (dado que la aplicación
    actualmente usa un procedimiento de ese tipo). Los algoritmos fueron
    puestos a competir a través de una serie de pruebas en una
    laptop estándar, usando matrices generadas del dominio de la aplicación. 
    Aun con un solver lineal optimizado tipo Cholesky, la implementación
    de Lanczos/IRLM más conocida (ARPACK) no pudo competir con LOBPCG,
    quién se vuelve el claro  
    ganador del experimento. Los resultados sugieren que para las matrices más
    grandes consideradas (size $\approx$ 4500), los tiempos de cálculo
    pueden moverse  
    de la escala de minutos a sub-segundos; y a un par de segundos, si
    agregamos el tiempo del pre-procesamiento para eliminar 
    clustered eigenvalues.  
