\chapter{The Algorithms}
\label{cha:algs}

\section{Proposal}
\label{sec:propo}
There is an immense quantity of algorithms out there, for solving the
Symmetric Eigenproblem. But by considering the symmetry, we can focus
on a considerably smaller subset of algorithms. As mentioned already,
the properties of the Laplacian drive 
our further pruning as it involves sparse and positive semi-definite 
matrices. \\

Our proposal is essentially to pick just three algorithms from that
immense tree and to evaluate their performance with a set of matrices
generated from the application domain. From such experiment, we will
pick the winner and hence, emit our official recommendation for the
application. While \cref{cha:exper} and \cref{cha:conclu} detail the
experiment and its outcome. This chapter focuses on explaining the
rationale behind this algorithms selection, along with a little
background about their theory and practical usage aspects. \\

Let us begin by saying that our intention with this selection was to
represent three families or approaches to the solution of the problem:
Current approach taken by the application (\gls{MRRR}), mainstream
recommendation for Spectral Clustering (\gls{IRLM}) and a modern
 proposal (\gls{LOBPCG}). In the sub-sections below we provide
further justification for each one of these algorithms.

\subsection{\gls{MRRR}}
Although the \gls{Laplacian} is a sparse matrix (70\% of zeros for our
target application), we still consider the family of algorithms that
care about dense matrix representations. This is mostly due comparison
purposes, as the current implementation the application uses, is
based on a method from this family. This will allow us to know how
much we improved the execution time with the new sparse matrix
algorithms. This is the argument behind the selection of the
\gls{MRRR} algorithm, which is not exactly the one used today by the
application. Thus, it can be considered as a lower bound in terms of its
performance. Whatever the application uses today is hardly better than
\gls{MRRR}, which is considered the ``holy grail'' of Numerical Linear
Algebra \cite{hogben06} (at least for dense matrices).\\ 

But the real search is within the sparse matrix sub-branch of
algorithms, and we justify our selection of \gls{IRLM} and
\gls{LOBPCG} in next two sub-sections.

\subsection{Lanczos: variant \gls{IRLM}}

Members of the Lanczos algorithms family, 
usually appear in Spectral Clustering literature as the mainstream
option to choose when computing the eigenpairs of the
\gls{Laplacian} matrix (see \cite{luxburg07} for example). Among them,
the Implicitly Restarted Lanczos Method (\gls{IRLM}), is perhaps
the most known and widely available variant, Thanks to the
ARPACK \cite{arpack} software package. This situation made us chose
\gls{IRLM} then, as it represents the common practice
against which we want to compare further algorithms (In reality, we are
also going to compare against a dense matrix algorithm).

\subsection{\gls{LOBPCG}}

The real proposal is to use \gls{LOBPCG}, which is a more modern
algorithm for solving the Symmetric Eigenproblem for sparse
matrices. The existing literature, see \cite{knyazev03}, suggest the
superiority of LOBPCG for scenarios like ours (computation of the
\gls{FiedlerVector}). 

\section{Preliminaries}

This section introduces the common mathematical tools used by the
selected algorithms.

\subsection{Power Method (for Symmetric Matrices)}

The most primitive algorithm for computing eigenvectors is the so
called Power Method. Although the slowest algorithm, it compensates
its limitations with its extremely simple definition. It can be
summarized by the two steps below (where $\vec{x}$ is the initial
approximation to the wanted eigenvector):

\begin{equation}
  \label{eq:power-method-iter}
    \vec{x}_0 = \vec{x} \ds{\ds{\land}}
    \vec{x}_{k+1} = A\vec{x}_k  
\end{equation}
\joinbelow{1cm}

The iteration from \cref{eq:power-method-iter}, surprisingly,
converges to the eigenvector 
associated to the biggest eigenvalue of the matrix $A$ (or to the one
associated with the smallest eigenvalue, if one uses $\inv{A}$ instead
\footnote{The presence of the inverse matrix $\inv{A}$ is mostly due
  notation, in practice we do not compute $\inv{A}x$, instead we solve the
  associated linear system $A\vec{y} = \vec{x}$.}). There is a
generic proof of this on \cite{saad92}, but a more accessible argument
is presented in \cref{eq:power-method-conv} for the particular case of
symmetric matrices. The vectors 
$\vec{v}_i$ are the eigenvectors of the matrix, and the key argument
for the development is that we
assumed the  matrix was symmetric, which implies its eigenvectors form a full
basis. This means any vector, including $x_0$, can be expressed as a
linear combination of them (see \cite{strang88} or \cite{golub13} for
more details). \\

\begin{align}
  \begin{split}
  \label{eq:power-method-conv}
  \vec{x}_k &= A^{k}\vec{x_0} \\
  &= A^{k}\left(\sum\limits_{\tiny{i=1}}^{n} \alpha_i\vec{v}_i \right) \\
  &= \sum\limits_{\tiny{i=1}}^{n} \alpha_i A^{k} \vec{v}_i  \\
  &= \sum\limits_{\tiny{i=1}}^{n} \alpha_i \lambda_i^{k} \vec{v}_i \\
  &= \alpha_n \lambda_n^{k} \vec{v}_n +
  \sum\limits_{\tiny{i=1}}^{n-1} \alpha_i \lambda_i^{k} \vec{v}_i \\
  & \overbrace{\approx}^{k \to \infty} \beta \vec{v}_n  
  \end{split}
\end{align}
\joinbelow{cm}


The Power Method does not always converge in exact
arithmethic \footnote{It requires that the initial vector has a non
  zero component coordinate regarding the dominant eigenvector}, but
here we focus on its 
more commonly found numeric limitation: slow convergence (the reader
can consult further details about the Power Method in \cite{golub13}
and \cite{parlett80}). This is
because it relies on the eigengap (distance between the desired eigenvalue and
the next one on that side of the spectra\footnote{Spectra is a
  commonly found name for the set of eigenvalues of a matrix.}). To be
more concrete, the convergence of the Power Method is proportional to
$\dfrac{\lambda_{n-1}}{\lambda_n}$ or $\dfrac{\lambda_1}{\lambda_2}$,
depending on whether we are seeking the  biggest or smallest
eigenvector. This limitation will appear
later, when comparing the algorithms.

\subsection{Krylov Subspaces}

The next tool is Krylov Subspaces (see \cite{parlett80} or
\cite{saad92}), which are used to search for 
approximations of the desired eigenvectors. A Krylov subspace of
dimension $m$, for a given matrix $A$ and generating vector
$\vec{x_0}$, is defined as follows: 

\begin{equation*}
    \Krylov{A}{\vec{x}_0}{m} =
    \func{span}
    \left\{
      A^0\vec{x}_0\ds{,} A^1\vec{x}_0\ds{,} \ldots \ds{,}  A^{(m-1)}\vec{x}_0
      \right\}  
\end{equation*}
\joinbelow{1cm}

Both \cite{parlett80} and \cite{saad92} have chapters with more
properties about Krylov Subspaces, but the intuitive idea is that the
Power Method wastes a lot of information: the iteration
produces the vectors $A^k\vec{x_0}$, but only the last one is actually
used. Krylov Subspaces keeps all these vectors, and use them to
generate a subspace where there are more chances to find good
approximations of the eigenvectors.

\subsection{Rayleigh-Ritz Method}

This is another way of extracting eigenvector approximations called
the Rayleigh-Ritz Method (see \cite{saad92}). Assuming
that you 
already have a subspace where you want to search for the
approximations of eigenvectors of matrix $A$, then the high level
algorithm looks like this: 

\begin{itemize}
  \item Compute orthonormal basis of that subspace, and arrange it as
    columns of a matrix called $V$.
  \item Solve the (smaller) eigenproblem $R\vec{y} = \lambda\vec{y}
    \ds{\suchthat} R = \trans{V} A V$. 
  \item Compute the Ritz pairs
      $(\apx{\lambda}_i,\apx{\vec{x}}_i) = (\lambda_i, V\vec{y}_i)$
\end{itemize}

As you can see on the above steps, in order for this method to work
the dimension of this subspace  needs to be much smaller than
$\func{dim}(A)$, because you are ultimately solving an small
eigenproblem anyway.  The rationale is that it becomes easier to
solve the eigenproblem on a much smaller dimension \footnote{As we
  will see later, algorithms targeting large sparse matrices can
  ultimately rely on dense matrix algorithms. This is with the premise that
  they do so at a much smaller scale.}. \\

Another interesting detail to note, is that the
matrix $R$ is something like a projection of the original matrix $A$
on the approximating subspace, the two matrices are pretty much the
same thing (similar) except for a change of 
basis. This implies the spectra of $R$ is a subset of the spectra of
$A$, so the eigenvalues obtained with Rayleigh-Ritz method can be used
directly as the desired approximations. The eigenvectors need a change
of basis though ($V\vec{y_i}$), which takes them from the coordinates
used on the approximating subspace to the canonical coordinates where
the matrix $A$ operates. More details about the theory behind this
method can be found in \cite{saad92}.

\section{Dense Matrix Algorithms}

As mentioned already, the only reason to consider this family of
algorithms is to have a reference point. But we just provide references for 
dense matrix algorithms in this section, as their details fall out of the scope of
this thesis (our focus is on sparse matrix methods). The canonical
encyclopedic reference for this family is \cite{golub13}, though each
algorithm may have more dedicated resources. We mention some on the
paragraphs below. \\

\subsection{Symmetric Tridiagonal QL Algorithm}
The application currently uses the Symmetric Tridiagonal QL
Algorithm (see \cite{parlett80}). Like other algorithms from same family, this method
converts first the input matrix into tridiagonal form, to
compute from there the eigenpairs. Once in this reduced form, the
simplest version of the algorithm could be thought as a block
version of the Power Method. On each iteration we need to
orthogonalize with some matrix factorization (QL) and to multiply the
resulting orthogonal vectors by the matrix (represented in a possibly
different basis). \\

Nevertheless, practical implementations have far
more points to consider. The reader interested in more details about
this QL Algorithms or its cousins the QR Algorithms, can consult
standard literature on the topic like \cite{golub13} or
\cite{parlett80}. Additionally we want to mention
here that this algorithm has an inherent limitation for our
application: It is designed to compute all the eigenpairs. This sounds
like a waste of resources, given that we just need one particular
eigenpair (the second smallest). 

\subsection{The \gls{MRRR} Algorithm}

The Multiple Relatively Robust Representations Algorithm (\gls{MRRR}
for short, or $MR^3$), is a more suitable option for computing the second
eigenpair \cite{dhillon97},
\cite{dhillon04}, \cite{dhillon06} or \cite{parlett04}. Is a quite
sophisticated procedure, but the only detail we will mention here, is
the ability of the algorithm to compute the eigenpairs in
isolation. Actually, this algorithm is perhaps the best 
the application can do, in terms of dense matrix methods.


\section{Sparse Matrix Algorithms}

If we really want to leverage the properties of the \gls{Laplacian}, we need
to consider algorithms which are suitable for sparse matrix
representations. They are usually designed to calculate an small
portion of the spectra (smallest or biggest), and the reason why they
work better with sparse formats is because they do not require
an explicit representation of the matrix at any moment. Instead, these
algorithms only require a callback mechanism to perform an operation
involving the matrix (eg $A\vec{x}$). Therefore, the internals of
those operations can be optimized to take advantage of the sparse
format in question (instead of using a generic logic that assumes a
dense representation). We will present a couple of algorithms from
this family: A variant of Lanczos and LOBPCG.

\subsection{Lanczos (\gls{IRLM})}
\label{sub:irlm}

Lanczos procedures are a family of algorithms, and the particular
variant we are presenting here is called Implicitly Restarted Lanczos
Method (\gls{IRLM}) \cite{arpack}. The algorithm is iterative (it
progressively approximates the 
desired eigenpairs), and the main idea is to apply the Rayleigh-Ritz
method against a Krylov subspace $\Krylov{A}{\vec{x_0}}{m} \suchthat m
> k$. This implies that it needs to solve on each iteration, an small
eigenproblem for an $m \times m$ symmetric matrix $H$. Where $m$ is a bit bigger
than the number of desired eigenpairs $k$, but still much smaller than the
dimension of the matrix. \\

One of the issues with Krylov Subspaces is that we do not know in
advance, how big the dimension $m$ needs to be in order to contain
interesting approximations to the sought eigenvectors. Since such
dimension is in function of the number of vectors $A^k \vec{x_0}$ that
we keep in memory, this can be a serious problem. In order to tackle
this issue, the \gls{IRLM} uses a tool called Implicitly Restarted QR
Algorithm to apply $p$ shifts against the matrix $H$:

\begin{equation*}
    j=1 \ldots p:\, QR = \func{qr}(H - \lambda_j I) \land H = \trans{Q}HQ
    \ds{\suchthat}
    m = k + p
\end{equation*}
\joinbelow{1cm}

After the shift a truncation with a factor $k$ takes place, and on
next iteration the matrix $H$ is filled up again to size $m$ (using
powers of the input matrix, among other stuff). 
At any iteration we have $m = k + p$ eigenpairs from matrix $H$, where
$k$ is the actual number requested by the user and $p$ is the number
of extra eigenpairs. The purpose of the shift procedure, is to 
discard the p vectors from $H$ that do not contribute to the
interesting eigenvectors. The definite reference, for all the gory
details of this variant of the Lanczos algorithm, can be found at
\cite{arpack}. \\

As far as practical usage of the \gls{IRLM} routine, we take the following
considerations:

\begin{itemize}
\item We set $k=2$ and by default the implementation sets $m=2k+1$.
\item As we want the smallest eigenpairs, we used shift-invert mode
  with $\sigma=0$ \footnote{Which means we 
  do not want the shift $A - \sigma I$, just the invert mode.}. This
  requires also to use a linear sparse solver. We tested two actually, \gls{SuperLU}
  (\cite{superlu97}, \cite{superlu05}) and  \gls{Cholmod} (\cite{cholmod08},
  \cite{cholmod08a}). These solvers use respectively the LU and
  Cholesky factorizations, hence are considered direct
  methods (as opposed to iterative linear solvers). While
  both are specialized for sparse matrices, only \gls{Cholmod} leverages
  the \gls{Laplacian} properties (we will see in the results chapter, how much
  that affects the performance). The linear solver we pass when invoking the
  routine in shift-invert mode, accounts for $\approx \%80$ of
  the execution time. Thus, is among the most important input parameters
  to consider. 
\end{itemize}

We observed slow convergence for \gls{ClusteredEigenvalues}, but this
was expected (is a known limitation inherited from the Power
Method, on which Krylov Subspaces are based on).


\subsection{\gls{LOBPCG}}
\label{sub:lobpcg}

The second algorithm being considered is the so called Locally Optimal
Block Preconditioned Conjugate Gradient (\gls{LOBPCG})
\cite{knyazev01}, \cite{lashuk07}. It is also an
iterative method that applies Rayleigh-Ritz Method every time (for
simplicity we focus on the single vector version). The main difference
with Lanczos is the subspace where it searches for the eigenvector
approximations, which for \gls{LOBPCG} is $\func{span} \{\vec{x}_i,
T\vec{r}_i, \vec{x}_{i-1}\}$. This is a fixed size set of 3 vectors,
containing respectively the current and previous approximations
($x_i$, $x_{i-1}$ \footnote{In practice $x_{i} - \beta x_{i-1}$ is used
  instead of the previous approximation $x_{i-1}$, as it tends to
  loose orthogonality respect to $x_i$.}) plus the preconditioned
residual $T\vec{r_i}$. The residual vector $r_i$ measures how well
$x_i$ approximates the desired eigenvector and is defined in
\cref{eq:lobpcg}. 

\begin{equation}
\label{eq:lobpcg}  
    \joinabove{0}
    \vec{r}_i = A\vec{x}_i - \func{\rho}(\vec{x}_i)\vec{x}_i
    \ds{\land}
    \func{\rho}(\vec{x}) = \dfrac{\trans{\vec{x}}A\vec{x}}{\trans{\vec{x}}\vec{x}}
\end{equation}
\joinbelow{1cm}

The function $\func{\rho}(\vec{x})$ is the already familiar 
Rayleigh-Quotient, and another of its properties justifies the
inclusion of the residual $r_i$. It turns out that
$\func{\rho}(\vec{x})$ has as critical point precisely the smallest
eigenpair $(\vec{v_1},\lambda_1)$. Furthermore, its gradient is
proportional to $r_i$. This means that $r_i$ is proportional to the
direction where the Rayleigh-Quotient Function approximates better
the smallest eigenpair. Hence, it makes sense to include such
direction when building the linear combinations that will approximate
the eigenpair. The details of this algorithm are quite
technical, but the reader can consult the original article from
its creator Knyazev (\cite{knyazev01}). An easier version
appears on the PhD thesis of one of Knyazev's students (see
\cite{lashuk07}).  \\ 

The reader may have noticed a little inconsistency. Above we talked about
the smallest eigenpair but was not the \gls{FiedlerVector} part of the
second one instead? That is still the case, but \gls{LOBPCG} has a nice
feature that allow us to focus on the smallest eigenpair indeed. When
invoking the routine, the user can pass a matrix $Y$ (called
constraints matrix) whose columns
generate certain subspace. Then the algorithm searches for the
approximations only in the orthogonal subspace $\ortc{Y}$. This fits
perfectly our problem, as we know that the first eigenvector of the
\gls{Laplacian} is the vector $\vec{1}$, and since we are dealing with a
symmetric matrix, we know that the rest of the eigenvectors (in
particular the \gls{FiedlerVector}) are going to live inside
$\ortc{\vec{1}}$. Thus, we set $Y = \vec{1}$, when calling the \gls{LOBPCG}
routine. This makes the search of the smallest eigenpair in
$\ortc{\vec{1}}$, equivalent to the search of the second smallest on
the original space. \\

Let us summarize now all the practical considerations taken when
calling the \gls{LOBPCG} implementation (which were actually grabbed from the
Python Package NetworkX \cite{networkx}):

\begin{itemize}
\item As mentioned, given we set the constraints matrix $Y = \vec{1}$,
  we only need to request the smallest eigenpair (set
  $k=1$). 
\item The matrix $T$ multiplying the residual vector $\vec{r_i}$ is called
  the preconditioner, and its purpose is to accelerate
  convergence \footnote{Going into the theory behind is out of
    our scope, but again,  the interested reader can consult \cite{knyazev01}
    or the more digested summary in \cite{lashuk07}.}. In practice, and for the
  data involved with our application, setting $T =
  \dfrac{1}{\func{diag}(L)}$ worked pretty well (actually, this is
  perhaps the most important input parameter for our case, without
  this preconditioner \gls{LOBPCG} was even slower than \gls{IRLM}). 
\item Rather than slow convergence, the implementation showed
  numerical errors on \gls{ClusteredEigenvalues} (for some reason, the
  dense $3 \times 3$ matrix produced at each iteration eventually
  looses the properties the code expects). We did not dig further into
  this problem, but just assumed that we needed to do something to
  avoid \gls{ClusteredEigenvalues} on the data, given that both sparse
  algorithms have problems with them.
\end{itemize}

