\chapter{Conclusions}

\section{Recommendations}

\subsection{Prefer Sparse Matrix Algorithms}

This is perhaps the most obvious recommendation that comes out of this
work; even if the data fits in memory (as it happens here), it makes
no sense to compute against zero values.

\subsection{If dense, go with MRRR}

If for some reason (like a prohibition to perform drastic code
chages), the application needs to go with dense matrices; at least
they could use the best algorithm available, which is MRRR. This can
be obtained from opensource implementations of LAPACK \footnote{There
  are also commercial, hardware optimized versions like Intel's; which
  could be considered.}, though Java
Native Interface (JNI) will be needed to call the routines from Java (which
is the applications' language). Alternatively, they can try the
Java-Netlib opensource project \cite{jnetlib}; which offers Java
wrappers against BLAS, LAPACK and other native libraries.

\subsection{If development time is a constraint, consider Lanczos}

If they can invest a bit of time (probably some weeks), and assuming
they are allowed to introduce more drastic changes like the usage of
sparse matrices, then Lanczos/IRLM is the way to go. Its ARPACK
implementation is also opensource, and also readily available through
the Java Netlib project. Of course the recommendation would be to use
the Cholmod linear solver, though that may add more time (in case JNI
wrappers do not exist yet).

\subsection{If speed is a concern, go with LOBPCG}

Finally, if they are willing to invest some months and if speed is the
main concern; then they can explore the winner of the competition,
LOBPCG. Porting the Python code from Scipy into Java \footnote{An
  interesting note about this code, is that the algorithm creator
  himself (Knyazev), helped with its development. This speaks about
  its quality.} may certainly be a non trivial task; but the impact
could be minimized if they restrict their attention to the single
vector version, which all we need to compute the Fiedler vector (the
actual code implements a block version that can approximate multiple
vectors, but we do not need that).

\section{Additional considerations}

\subsection{Need sparse format}

As explained in previous chapters, for the sake of the algorithms we
could have used either CSR or CSC formats; but the need to compute the
SCC of the graph behind the Laplacian, make us pick CSR
format. If they are going with either Lanczos or LOBPCG, this is the
format we recommend to use. \\

The format may not come for free though; the application currently
uses a serial version of the Colt library \cite{colt}, which does not
seem to offer any sparse format. Thus, additional effort will be
needed to research which libraries support CSR in Java. \\

Let us recall the sparse algorithms will not care if the caller uses
CSR or not; as they do not require explicit representation of the
matrix. However, while implementing the callback mechanisms that do
compute on the Laplacian (either $L\vec{x}$ or solving $L\vec{y} =
\vec{x}$), we should definite use the sparse format; otherwise the
promised gains will not show up.

\subsection{Algorithm for computing the SCC}

The algorithm used for efficiently computing the SCC, along with the
code to recalculate the weights matrix; is all in Python (probably
using other native libraries). Assuming we manage to make the CSR
format available to Java, another task to include will be to look for
a port of this algorithm (or to implement it using the article
\cite{pearce05}). While this task may look non trivial, at least does
not require deep knowledge of Numerical Analysis; which should make it
at least tractable by regular programmers.

\section{Challenges found during this thesis}

Finally, we would like to dedicate a few lines about some challenges
that were found during the elaboration of this 6 months thesis
project.

\subsection{Numerical Linear Algebra is hard}

For the non initiated, meaning students who have not taken advanced
courses in Linear Algebra,  Numerical Linear Algebra and Numerical
Analysis in general; is quite a challenge to grasp the algorithms
explanations found in literature. There seems to be a tradition to
assume that the reader has this immense background, as many details
are omitted. Therefore, one needs to go back to more elementary
material to try filling some of the gaps. This makes the overall
progress a bit slow, and that is the reason why only 3 algorithms were
presented (more on this below). \\

We dared to embrace this topic thinking naively that a previous
course project would provide the required context; but this was
true just partially. The specialization required to fully grasp the
theory behind the 3 algorithms compared in this work, goes far beyond
the context acquired during the last year. That is why we reduced the
scope to understand only the main ideas, and to learn how to use in
practice the implementations.


\subsection{There is zoo of algorithms out there}

The 3 algorithms compared in this thesis, definitely do not represent
the entire set of options available. Even if we restrict ourselves to
the properties of the Laplacian, and to serial execution; there are
several more that offer an opensource implementations to
explore. Just to give an idea of the diversity available online, the
following is a partial list of additional options we also considered
(meaning that we read a bit about the algorithms, that we installed
the software and tried at least once):

\begin{itemize}
  \item Anasazi: Software for the numerical solution of large-scale
    eigenvalue problems \cite{anasazi}.
  \item Slepc: Scalable and flexible toolkit for the solution of
    eigenvalue problems \cite{slepc}.
  \item Primme: preconditioned iterative multimethod
    eigensolver-methods and software description \cite{primme}.
  \item Lanczos/IRLM from ARPACK, but using iterative linear solvers
    (like those reported in \cite{martinez16}). 
\end{itemize}

All the options from the above list are quite interesting to explore, 
along with solid publications about their theoretical
fundations. Except for the last one (for which we did not find an
available implementation), the rest were tried indeed; but we just did
not have enough time to research more about their inner workings, nor
to understand how to use the in practice to compute the Fiedler
Vector. If more time becomes available, this would be an interesting
follow up for this thesis, to explore the options listed above.

